\section{Introduction}
%    - What does polarity mean in lactiferous ducts?
%    - What macromolecules are involved?
%    - What processes does polarity regulate?

Loss of cellular organisation and polarity is a common feature across epithelial cancers, but unlike some other cancers like those that present in the colon where cell polarity is lost at late stages of the disease, loss of polarity is a hallmark of early breast cancer \citep{hinck2014}. The mechanisms by which epithelial cell polarity is lost in carcinomas, however, remains elusive and poorly understood.\par

\subsection{The Role of Protein Complexes in Ductal Polarity \& Cancer}

%        - [*] Crumbs complex proteins
%        - [ ] Crumbs3 & TJs (PATJ/Pals1)
%        - [*] FERM domain partners
%            - [*] Expandable/FRMD6
%            - [*] Yurt/Ehm2
%        - [*] Density sensing & Hippo (Not via FERM proteins)
%        - [ ] Crb3 homophilic interactions
%        - [ ] Pals1/Par6/aPKC & migration direction
%            - [ ] http://embor.embopress.org.proxy3.library.mcgill.ca/content/8/2/158
%            - [ ] http://www.sciencedirect.com/science/article/pii/S0092867402012497
%        - [ ] ZEB/Snail/Crb3 and EMT
%        - [ ] Crb3 homologues and isoforms?
%
% READ-UP ON STATEMENT:
% Might be tie-in to Ruba's paper?
% "The Ehm2/p114RhoGEF module organizes the circumferential actomyosin belt by
% activating RhoA and its effector kinases ROCK1 and ROCK2 (ROCK1/2)."
% doi: 10.1128/MCB.00673-15
%
% Nakajima H, Tanoue T. 2010. Epithelial cell shape is regulated by Lulu proteins
% via myosin-II. J Cell Sci 123: 555–566. http://dx.doi.org/10.1242/jcs.057752.
%
% Nakajima H, Tanoue T. 2012. The circumferential actomyosin belt in epithelial
% cells is regulated by the Lulu2-p114RhoGEF system. Small GTPases 3: 91–96.
% http://dx.doi.org/10.4161/sgtp.19112
%
% Terry SJ, Zihni C, Elbediwy A, Vitiello E, Leefa Chong San IV, Balda MS,
% Matter K. 2011. Spatially restricted activation of RhoA signalling at epithelial
% junctions by p114RhoGEF drives junction formation and morphogenesis.
% Nat Cell Biol 13:159–166. http://dx.doi.org/10.1038/ncb2156
%
% Nakajima H, Tanoue T. 2011. Lulu2 regulates the circumferential acto-myosin
% tensile system in epithelial cells through p114RhoGEF. J Cell Biol 195: 245–261.
% http://dx.doi.org/10.1083/jcb.201104118

When discussing the epithelial polarity of the lactiferous duct, one may be referring to asymmetric distribution at either the intercellular or intracellular level.\par

At the macro, intercellular scale, ductal epithelia is said to exhibit tissue
polarity when cells organise into a monolayer forming a single lumen \citep{bissell2003}.\par

On the other hand, establishment of cellular apical-basolateral polarity is
achieved by the intracellular asymmetric distribution of proteins, phospholipids
, and carbohydrates within the inner epithelial monolayer of the mammary duct.
Two protein complexes, the Crumbs complex and the Par complex, are particularly
important determinants of the apical identity \citep{horikoshi2009,whiteman2014}.\par

\subsubsection{The Crumbs Complex}
Localisation of the Crumbs complex to the plasma membrane both contributes to
the establishment and maintenance of its apical identity. The Crumbs complex
converges upon the apical transmembrane glycoprotein for which it is named,
Crumbs3 ({\it Crb3}), which serves as a scaffold for the complex. Crumbs3 directly
binds two proteins via its carboxy-terminal PDZ domain ({\tt ERLI}): protein
associated with {\it Lin-7} one (Pals1) and partitioning-defective protein six (Par6) \citep{lemmers2004, roh2002}.The presence of Pals1 also brings to the Crumbs complex the Pals1 associated tight junction (PATJ) protein, which is essential for proper polarisation and contributes to the establishment of tight-junctions in mammalian cells \citep{shin2005}.\par

% Insert Crb3 and TJs...

Crumbs3 has also been known to interact with FERM (4.1 protein, ezrin,
radixin and moesin) domain proteins through its PDZ domain. Crumbs3 also
interacts with the FERM-domain proteins EHM2 (also known as Lulu2) and YMO1,
homologues of {\it Drosophila melanogastor} protein Yurt, which helps to establish
apical-basolateral polarity and maintain the size of the apical membrane by
regulating Crumbs3 \citep{laprise2006}. Crumbs3 has also
been shown to recruit EHM2 and p114RhoGEF to maintain the actomyosin belt and
promote cell-cell adhesion in a cancer cell-lines, requiring both the C-terminal
FERM-binding and PDZ-binding motifs of Crumbs3 \citep{loie2015}.\par

The crumbs complex has been also shown to regulate important proliferative
programmes such as organ growth and mammary gland contact inhibition through the
Salvador/Warts/Hippo (hereafter Hippo) signalling pathway. Crumbs3 regulates the
Hippo pathway through interactions with, among other proteins, the FERM
domain-containing protein 6 ({\it FRMD6}), a mammalian homologue of the
{\it D. melanogastor} gene {\it Ex} \citep{robinson2010}. Crumbs3
also regulates the Hippo pathway through direct interaction with WW-domain proteins.
One such instance is the direct interaction between the Hippo pathway co-effectors
yes-associated protein 1 (YAP1), Tafazzin (TAZ), and Crumbs3; this occurs in
response to changes in cell density, which require changes to the cells
proliferative program \citep{varelas2010,szymaniak2015}. In a similar, cell-density-sensing
manner, Crumbs3 interacts directly with Kibra's WW-domain to stabilise it,
preventing its degradation and promoting Hippo-pathway-mediated proliferation
\citep{moleirinho2013,mao2017}.\par

Cell density is also coupled with transforming growth factor-$\beta$ (TGF-$\beta$)-induced
epithelial-mesenchymal transition (EMT) through the Crumbs3-mediated inhibition
of SMAD; effectively reducing downstream activation Snail
\citep{varelas2010}. In addition to being an important to the
EMT transcriptional programme, the zinc-finger protein Snail ({\it SNAI1}) is a potent
transcriptional inhibitor of Crumbs3 and to a lesser extent, PATJ and PALS1;
resulting in mislocalisation of the Crumbs and Par complexes and disruption of
tight-junction and polarity formation \citep{wang2013, whiteman2014}.\par

\subsubsection{The Par Complex}
The Par complex is named after the eponymous family of proteins first
discovered in the 1980s as part of screen to identify maternal effect lethal
mutations in the model nematode {\it Caenorhabditis elegans}
\citep{kemphues1988,goldstein2007}.
Of the six, the par proteins most relevant to apical membrane specification are
Par3 and Par6; both PDZ-domain scaffolding proteins and core members of the Par
complex \citep{yu2014,hung1999}. Other
members of the Par complex include atypical protein kinase C (aPKC) and the
cell division control protein 42 homologue (Cdc42) GTPase, binding to Par3
through Par6 which here acts as an adaptor \citep{joberty2000}.
Par3 is also capable of binding aPKC directly; an interaction that is essential
to establishing proper cell polarity, normal ductal architecture, and mammary
gland morphogenesis \citep{nagai2002, mccaffrey2009}.\par

The formation of the Par complex is cued by the establishment of cell-cell
contacts. This comes as a result of the complex being anchored to the tight-
junctions of the cell by Par3, which is tethered through its binding of
phosphotidyl inositols and the junctional adhesion molecule (JAM) through the
second and first of Par3s three PDZ domains, respectively \citep{wu2007,ebnet2001}.\par

At their basal levels of expression, the members of the Par complex are at a
regulatory equilibrium that is often disrupted in neoplasias, resulting in
aberrant signal integration and epithelial disorganisation. Human breast
cancers often express dramatically reduced levels of Par3, freeing aPKC to
inappropriately activate signalling pathways that lead to increased invasive and
metastatic potential; namely the human epidermal growth factor receptor 2 (HER2)
and janus kinase/two Signal Transducer and Activator of Transcription (JAK/STAT)
pathways \citep{xue2013,mccaffrey2012}.\par

aPKC is but one member of the mammalian Protein Kinase C (PKC) super family, of
which there are three additional members: the classical/conventional PKCs
(cPKCs), the novel PKCs (nPKCs), and the later-discovered PKC-related kinases
(PRKs). These families are distinguished by their dependence/independence
on $\textrm{Ca}^{2+}$, and whether they are activated by diacylglycerols (DAGs). aPKC is both $\textrm{Ca}^{2+}$-independent and DAG-insensitive, and in this respect are identical to PRKs.
The two families are primarily differentiated by PRKs association with RhoA,
which is unique among the PKCs \citep{mellor1998}.\par

Each PKC sub-family contains multiple isoforms, which each confer their own
unique function. The aPKC exists in two isoforms, aPKC$\lambda$/$\iota$ and aPKC$\zeta$.\par
