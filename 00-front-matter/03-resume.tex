\newpage
\section*{Résumé}
\addcontentsline{toc}{section}{Résumé}

Un signe avant-coureur d’une maladie beaucoup plus difficile à traiter, 
le cancer du sein précoce est un sujet qui est important à étudier, mais 
qui est souvent négligé. Il est souvent difficile à détecter les cancers 
du sein précoce et pour les classifier. Les méthodes histopathologiques
utilisées pour détecter les cancers du sein précoce sont très subjectives
et résulte dans les diagnostics variables entre les cliniciennes et 
les cliniciens. C'est possible de mitiger ces problèmes avec les systèmes
informatiques quantitatifs, mais celles qui existe néglige de prendre
en compte l'hétérogénéité intratumorale. C'est aussi souvent le cas que 
ces systèmes informatiques fonctionnent dans une manière obscure et inconnue
à leurs utilisateurs. Ces limitations empêchent l'utilisation de ces systèmes
dans les contextes cliniques.

Nous vous présentons deux systèmes informatiques qui étaient développés en  
réponse du problème de l'hétérogénéité intratumorale. Ceci est accompli par  
l'annotation des régions dans les champs de microscopies qui démontre des 
sections des biopsies mammaires qui semblent d'appartenir à une tumeur à un
stade précoce. Le système {\guillemotleft PPReCOGG\guillemotright} est 
un classificateur accéléré par la GPU qui utilise les textures visuelles 
pour identifier et annoter (avec haute résolution) des régions qui démontre 
une configuration de cellules caractéristique des cancers précoces, tout 
avec une précision très robuste (une précision moyenne de $\approx94{,}3\;\%$ 
sur les essais synthétiques). {\guillemotleft DeepDuct\guillemotright} est 
une système d'apprentissage automatique qui classifie et localise des tumeurs
dans un façons précise et transparent en utilisant l'algorithm {\guillemotleft Grad-CAM\guillemotright}. Ces deux systèmes démontrent que c'est possible de 
produire des classificateurs automatiques pour le cancer de sein
qui sont utilisable dans un contexte clinique grâce à leurs hautes précisions, 
capacité pour interpréter l'hétérogénéité intratumorale, est pour tous faire
dans une manière transparente pour l'utilisateur.