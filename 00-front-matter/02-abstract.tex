\newpage
\section*{Abstract}
\addcontentsline{toc}{section}{Abstract}

The harbinger of a far more lethal and difficult-to-treat disease, 
early breast cancer is a key, but often overlooked, point of study. 
Early breast cancers are often difficult to detect and classify, and 
the subjective nature of the histopathological methods used to detect 
them often result in dissenting diagnoses between clinicians. These 
issues can be mitigated through the use of quantitative computational 
methods, although existing solutions do not account for the 
heterogeneity of tumours, and often do not operate in manner that is
transparent to the clinician; precluding their utility in a clinical 
setting.

This work presents two models which address the concerns of tumour 
intraheterogeneity by detecting, classifying and annotating regions 
within fields of breast biopsy sections that appear to belong to early 
lesions. The PPReCOGG model, which is a GPU-accelerated texture-based 
classifier, is able to generate pixel-resolution annotations of cell-patterning
that is characteristic of early lesions with robust accuracy 
($\approx94.3\%$ average accuracy in synthetic benchmarks). DeepDuct 
is a deep learning model that provides accurate and transparent 
localisation and classification of lesions using gradient-based class 
activation maps (Grad-CAM). These two models illustrate that it is 
possible to develop clinically relevant classifiers that can achieve 
robust accuracy and account for tumour heterogeneity and model 
transparency.