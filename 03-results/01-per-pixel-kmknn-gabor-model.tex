\section{Machine-Learning-Based Regional Early-Lesion Recognition}

Whole-image machine-learning-based classifiers of early breast lesions
from microscopy images of breast biopsies with various accuracy rates have been
previously described. While these models are effective at detecting which stage
is most represented within a lesion, this poorly reflects the heterogenous nature
of breast cancers. Furthermore, whole-image classifiers offer little information
when presented with so-called ``borderline'' lesions, that exhibit
characteristics of many early lesions in near-equal measure.\par

A model capable of classifying regions of a given image of a lesion would
offer an added dimension of information that takes into account the heterogenous
nature of cancers, while also providing reproducible, quantifiable insight into
currently enigmatic "`borderline`" lesions.\par

\subsection{Per-Pixel Gabor-Filter-Based Classification kMKNN Model of Early Breast Lesions}
In a first attempt at developing such a classifier, a kMKNN model was trained on
per-pixel Gabor-filter features of confocal imagery of mammary glands
immunohistochemically stained for E-cadherin and Par-6. This model will be
referred to as the Gabor-kMKNN model hereafter.\par

As the characteristic ``cobblestone'' epithelial phenotype is altered distinctly
in both ADH and DCIS, it was hypothesised that a model trained on texture-based
features would be sufficient in classifying those lesions. For this reason, the
Gabor-filter was chosen. Analagous to the manner in which simple cortical cells
perceive patterns and texture, the Gabor filter and has long been used to
programmatically discern textures from one another
(Fogel \& Sagi 1989, Marĉelja 1980).\par

Gabor features were extracted in a similar fashion as Melendez \emph{et al.} 2008 and is described by \emph{Figure 1}. Namely, for each pixel in an image, six windows of increasing size ($3\times3$, $5\times5$, $9\times9$, $17\times17$, $33\times33$, $65\times65$) centered on the pixel are defined. Each window is then filtered through four Gabor kernels with quarter-turn orientations (\emph{i.e.}: $\theta = \left\{\frac{1}{2}\pi, \pi, \frac{3}{2}\pi, 2\pi \right\}$). Each Gabor kernel also has a sinusoidal wavelength of 0.25 pixels ($\lambda = 0.25$), which has been previously described as providing good discrimination in general-purpose texture classification
(Manjunath \& Ma 1996).\par

%figure
%Diagram illustrating the method by which Gabor features are extracted from images in the Gabor-kMKNN model

The mean and the standard deviation of the resulting Gabor energies are then
added to a vector for the relevant pixel. This results in a total of 48 features
per pixel (6 windows $\times$ 4 orientations $\times$ [1 mean + 1 standard deviation] = 48
feature per pixel).\par

To reduce computational complexity, images are resized to a resolution of
$256\times256$ pixels. Computation time of training can be further reduced by
extracting the features of a smaller randomly sampling of pixels from the
total population.

Following the $k$-means for $k$-nearest neighbor
($k$M$k$NN) model described by Wang 2011, the resultant 48-dimensional
matrix is clustered using the $k$-means algorithm. The number of
clusters is determined using the heuristic:

$$
k_c = \left \lceil{2\sqrt(n)}\right \rceil
$$

Where $k_c$ is the number of clusters to compute, and $n$
is the number of elements to cluster. If features were computed for all 65,536
pixels in a $256\times256$ image, a total of 512 clusters would be computed
$\left(k_c = \left \lceil{2\sqrt(65,536)}\right \rceil = 512\right)$.

The process of classifying a pixel ($p$) begins by calculating the 48 Gabor
energy means and standard deviations of $p$ as described above. The nearest
cluster ($C'$) is determined by choosing the cluster with the minimum Euclidean
distances between its centroid and the feature vector of $p$. $p$ is then
classified using the standard $K$-nearest neighbor algorithm with
feature vectors contained in $C'$ and a $K$-value of 3.

While the Gabor-kMKNN model was efficient at identifying distinct textures in
images composed of multiple MeasTex
