%% NOTE
%% The current practices section used to be in the intro to one of the ML chapters, but I:
%% (1) don't want to make a redundant section in the second ML chapter, but 
%% (2) am not sure if it's sufficiently global enough for the intro or 
%% (3) is too "far away" from the ML sections

\section{Current Practices for the Diagnosis of Early Breast Cancers}
Patients which exhibit symptoms of mammary neoplasmas, or are otherwise suspected to be at elevated risk due to factors such as age or family history, are subjected to routine screening by mammography. The sensitivity with which mammography is able to detect cancerous lesions, especially in young patients or patients whose breast tissue\footnote{In the former group, the distinction between invasive and pre-invasive lesions is often unclear, while the dense tissue of the latter group can obscure and mask possible lesions} \citep{ayvaci2014}. In such cases, screening via ultrasound can be a viable alternative, as it offers greater sensitivity \citep{nothacker2009}.

Suspected lesions detected during the process of screening are diagnosed by histopathological analysis of tissue biopsies \citep{nccn}. Tissue obtained through biopsy (core-needle, surgical, or otherwise) are sectioned onto glass slides, fixed and stained with relevant histological stains such as hematoxilin and eosin (H\&E).

%% NOTE
%% Possibly transition into molecular subtypes and their prognostic value. 
%% Possibly further transition in to current attempts to predict molecular subtype/outcomes.

Despite the standardised and continually refined methods and guidelines clinical pathologists rely on to identify breast cancer lesions from histological sections, there is a great deal of inconsistency and uncertainty that is becoming increasingly apparent. While some types of breast cancer (such as high-grade DCIS and LCIS) are more consistently and reliably identified than others, inter-observer agreement between clinical pathologists is mixed. In a retrospective study, agreement between pathologists for ADH, FEA, and low-grade DCIS regions was only moderate (0.44, 0.47, and 0.47 Cohen's $\kappa$ statistic, respectively) \citep{gomes2014}.\par
