\section{Automated Detection and Classification of Early Breast Lesions}

\subsection{Current Practices for the Diagnosis of Breast Cancers}

The current standard of care for the diagnosis of breast cancer is the histopathological analysis of tissue biopsies \citep{nccn}. Sections from biopsy tissue are routinely stained with hematoxilin and eosin (H\&E) and immunohistochemical studies performed to detect the presence of the HER2 oncogene and the estrogen and progesterone receptors \citep{who_intro}.\par

Despite the standardised and continually refined methods and guidelines clinical pathologists rely on to identify breast cancer lesions from histological sections, there is a great deal of inconsistency and uncertainty that is becoming increasingly apparent. While some types of breast cancer (such as high-grade DCIS and LCIS) are more consistently and reliably identified than others, inter-observer agreement between clinical pathologists is mixed. In a retrospective study, agreement between pathologists for ADH, FEA, and low-grade DCIS regions was only moderate (0.44, 0.47, and 0.47 Cohen's $\kappa$ statistic, respectively) \citep{gomes2014}.\par

\subsection{Conventional Machine-Learning Models for the Diagnosis of Breast Cancers}

While computer-aided detection (CADe) is sometimes used to enhance interpretation of imaging from screening mammographies, computational models are not currently used in most clinical settings to assist the diagnosis of breast cancers.\par

Models using conventional supervised machine-learning (ML) algorithms, such as support-vector machines (SVMs) or random decision forests (RDFs), have been previously described with varying success rates \citep{anuranjeeta2017, gertych2015}. These models depend on automatically extracted features hand-picked by their designers. The features chosen to train such models are directly related to existing pathological guidelines (\emph{e.g.}: nuclear size and spacing) or inferred by them (\emph{e.g.}: texture-based features such as Gabor filters and Harlick transforms) \citep{anuranjeeta2017, doyle2008}.\par

\subsection{Convolutional Neural Network Models for the Diagnosis of Breast Cancers}

Convolutional Neural Networks (CNNs or ConvNet) are deep machine learning algorithms that use multiple weighted hidden layers of convolutional filters to make decisions about a given input. Recent developments in general-purpose computing on graphics processing units (GPGPU) have made the use of ConvNets practical, and as a result ConvNets have since been shown to be particularly well suited for the task of classifying and analysing images \citep{ciresan2011, ciresan2012}. Neural networks have the advantage of dynamically ``learning'' and optimising features, as opposed to relying on human-selected features. This manner of feature selection allows for the potential discovery of new underlying concepts that fundamentally define a class from others, and also prevents assumptions and misconceptions from biasing features and introducing inaccuracies.\par

%NOTE: Might be better suited for Discussion section
The automated nature of feature selection in neural networks can also pose challenges if the dataset is of low-quality or very small; as differences between classes that are present in the dataset, but do not reflect differences between the classes as a concept, can severely bias a model. This is epitomised by ``the tank problem'', whereby a neural network trained by the US government to identify concealed tanks were actually identifying dark skies as the images of tanks in the dataset were taken on a cloudy day \citep{dreyfus1992}. Errors such as these are mitigated by assuring datasets are large and representative of the variation of the population.\par

ConvNets have been successfully used to create very accurate models for the classification of breast cancer lesions. Binary models for classifying benign and malignant lesions, as well as multi-class models for distinguishing between multiple subtypes of breast lesions from H\&E stained biopsy slides have established with very high ($>90\%$) accuracy \citep{wei2017, han2017}.  


%%\subsection{Regional Classification Models}

%%It