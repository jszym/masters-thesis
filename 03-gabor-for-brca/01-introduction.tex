\section{Introduction}

When diagnosing breast cancer lesions, trained clinical pathologists rely on 
visual interpretation of patient biopsies. While interobserver variability 
between trained pathologists is a known problem, it is particularly pronounced 
when pathologists are called to make distinction between early breast cancer 
lesion stages \citep{gomes2014}.\par

Computer-aided detection (CADe) is sometimes used to enhance 
interpretation of imaging from screening mammographies and can help manage the ambiguous and subjective nature of medical imaging; however computational models 
are not currently used in most clinical settings to assist the diagnosis of breast 
cancers.\par

Despite many being described in the literature, a major limitation of most breast 
cancer classification models is that they fail to model intra-tumour heterogeneity 
as they commonly adopt whole-field classification modalities
\citep{pareja2017,weigelt2010}. Whole-field classification models similarly do
not offer insight into so-called ``borderline'', which are lesions containing regions exhibiting features of multiple early lesions \citep{masood2011}.\par

To account for borderline lesions and the heterogeneous nature of cancer, a practical model for the classification of early lesions would be required to classify sub-regions of whole sections. As the characteristic ``cobblestone'' epithelial phenotype is altered distinctly
in the early progression of breast cancers, such a model would benefit from leveraging features which are based on differences in cell patterning.\par

The observation that early lesions exhibit distinct cell patterning unique to themselves are reported in the first descriptions of hyperplasia and carcinoma \textit{in situ} of both the mammary duct and lobules by \cite{page1982}. The descriptions provided of the early lesion are strongly based on cellular architecture and patterning; distinguishing, for example, ductal and lobular carcinomas \emph{in situ} (DCIS and LCIS) from atypical ductal and lobular hyperplasias  (ADH and ALH) by ``...round, regular spacing''  in the former and their absence in the latter, sometimes exhibiting ``...swirls or streaming''.\par

With the aim of developing a model that classifies regions of early neoplasms from whole sections of breast biopsy according to aberrations in their cell patterning, the authors here present a model for the per-pixel recognition of cancers
using oriented Gabor filters on the GPU (henceforth referred to by the acronym 
PPReCOGG). Analogous to the manner in which simple cortical cells perceive 
patterns and texture, the Gabor filter and has long been used to 
programmatically discern textures from one another 
\citep{fogel1989, marcelja1980}. Its purpose in this model is as a 
texture-dependent feature that differs between early cancer lesions according
to their distinct epithelial patterning.\par

The PPReCOGG model achieves a high rate of accuracy with an
average of $\approx94.3\%$ of pixels being correctly classified on synthetic 
validation classification tasks, and also is demonstrated to effectively identify sub-regions exhibiting characteristic neoplastic cell patterning in images of human early breast lesions.

%The use of per-pixel texture analysis in PPReCOGG directly address these issues by annotating sub-regions, much like a trained pathologist would, of differing early stages.


%%%%%%%%%%%%%%%%%
%Whole-image machine-learning-based classifiers of early breast lesions
%from microscopy images of breast biopsies with various accuracy rates have been
%previously described. While these models are effective at detecting which stage
%is most represented within a lesion, this poorly reflects the heterogenous nature
%of breast cancers. Furthermore, whole-image classifiers offer little information
%when presented with so-called ``borderline'' lesions, that exhibit
%characteristics of many early lesions in near-equal measure.\par

%A model capable of classifying regions of a given image of a lesion would
%offer an added dimension of information that takes into account the heterogenous
%nature of cancers, while also providing reproducible, quantifiable insight into
%currently enigmatic "`borderline`" lesions.\par