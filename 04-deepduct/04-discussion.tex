\newpage
\section{Discussion}

% Summary of what you've done
The DeepDuct model described herein provides a proof-of-concept framework for the localisation of breast lesions from H\&E staining that does not rely on manual feature selection, transparently reports explanations for its predictions via class activation maps and allows for the potential discovery of new features that could inform future manual diagnosis.\par

% Why it's important
Neural networks have the advantage of dynamically ``learning'' and optimising features, as opposed to relying on a manual process of feature engineering that is often driven by limited powers of intuition and conventional knowledge. The automated process of feature engineering allows for the potential discovery of new underlying concepts previously not described in the literature that fundamentally define a class from others, and also prevents assumptions and misconceptions from biasing features and the resulting inaccuracies.\par

%Realising the impact that opaqueness has on the practicality of classifiers in these critical contexts, solutions have been created to provide insight and understanding regarding how models have arrived at their predictions, such as LIME and Grad-CAM \citep{}. To the author's understanding, DeepDuct is the first application of such technology to the classification and localisation of breast lesions from H\&E.
\subsection{Explanations for Neural Network Predictions are Essential for Clinical Use of CADe/x Models}

% How it stacks up to things other people have done
Whole-image ConvNet classification models fine-tuned on the BreakHis dataset have been previously described reporting high-accuracy, as well as patch-based whole-slide classifiers which apply whole-image classification to small patches of an imaged slide resulting, in a form of tumour localisation \citep{han2017, wang2016}. None of these models, however, offer the same extent of transparency and resolution offered by the activation maps provided by the Grad-CAM algorithm used in the DeepDuct model. Existing models of breast lesion classification and localisation remain ``black-box'' solutions to end-users, particularly those without in-depth knowledge of deep learning algorithms.\par

Unique among deep-learning based breast lesion classifiers, DeepDuct reports which regions of the input image have lead the model to classify the image as it had. This simultaneously allows for general localisation of classified objects and a glimpse into the internal state of the model, informing and not prescribing a diagnosis. This transparency is essential for a model to see use in contexts such as clinical settings where acting on predictions in blind faith is not an option due to the high-risk associated with the decisions being made. Models that implement ``explanations'' for their predictions have indeed been shown to increase end-user trust in model predictions, as well as help identify false-positive predictions made by a given model \citep{ribeiro2016}. To the author's understanding, DeepDuct is the first application of such explanatory algorithms to the classification and localisation of breast lesions from medical imaging.\par

% It's limitations
\subsection{Dataset Considerations}
\hl{subsection-needs-better-name}\\
The BreakHis dataset, while covering a number of relevant lesion types with a significantly large number of examples for each type, presents some important challenges. Firstly, despite the multiple magnifications provided, images in the BreakHis dataset are not of high-resolution, taken with a digital camera with pixel size of 6.5$\mathrm{\mu m}$ and resolution of 480 TV lines. Secondly, the number of examples is extremely imbalanced between classes, with as much of a 7.5-fold difference between the least represented class and the most represented class.\par

While low-resolution images can be useful for learning so-called ``global features'', they've proven to be problematic when distinguishing differences between objects with similar high-level features, as is the case between two H\&E images exhibiting different lesion subtypes. This problem is illustrated well in the description of Baidu's Deep Image model, whereby similar objects (such as insects of the same species) can only be distinguished from one another when higher-resolution images are considered in the model \citep{baidudeepimage}. Training the DeepDuct model on higher resolution datasets would address this concern. High-resolution datasets of breast lesions do exist, but many offer too few examples (INESCTC) or do not offer histological type information outside grade (CAMELYON16) \hl{citation-needed}.

As described early, imbalances in the number of examples provided per class in the BreakHis dataset had lead to a strong bias towards over-represented classes (\textit{Figure \ref{fig:confmat}a}). This bias was addressed by oversampling all under-represented classes by duplicating examples until all classes in the training set contained the same number of examples (\textit{Figure \ref{fig:confmat}b}).

\subsection{Future Directions \& Improvements}

Implementing the DeepDuct model on smartphones would afford clinicians low-cost, mobile tools for the annotated classification of breast histology slides through use of commercial or 3D-printable smartphone-microscope adapters \citep{cellphone_microscope_platform}. A less computationally-complex mobile DeepDuct implementation would be required to account for the limited resources available on the platform. This is typically achieved either by a networked server-client model supported by computation in the cloud, or by replacing the deep, resource-heavy VGG16 model with a more shallow mobile-oriented model such as SqueezeNet \citep{squeezenet}. While the former is limited by patient-privacy compliance and network connectivity, the latter requires retraining the network on a shallower ConvNet architecture with potential losses in accuracy.

Regional convolutional neural networks (R-CNNs), such as Facebook's Mask R-CNN, have been developed to provide pixel-resolution object detection in complex scenes \citep{mask_r_cnn}. Using a Mask R-CNN model trained on a breast lesion dataset can provide higher resolution lesion detection than existing patch-based breast lesion models \citep{wang2016}.

%The DeepDuct algorithm might also be combined with regional convolutional neural network (R-CNN) algorithms such as Mask to provide both explanations (in the form of Grad-CAM heatmaps) as well as per-pixel 

%Limited computational resources reduce the practicality of CADe/x such as DeepDuct which are based on convolutional neural networks. Reducing computational complexity can be achieved by using smaller, mobile-oriented architectures such as SqueezeNet.

% Improvements/Future Directions

%The automated nature of feature selection in neural networks can also pose challenges if the dataset is of low-quality or very small; as differences between classes that are present in the dataset, but do not reflect differences between the classes as a concept, can severely bias a model. This is epitomised by the parable of ``the tank problem'', whereby a neural network trained by the US government to identify concealed tanks was actually identifying dark skies as the images of tanks in the dataset were taken on a cloudy day \citep{dreyfus1992}. Errors such as these are mitigated by assuring datasets are large and representative of the variation of the population. \par

 