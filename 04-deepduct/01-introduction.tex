\section{Introduction}

\subsection{Convolutional Neural Network Models for the Diagnosis of Breast Cancers}

Convolutional Neural Networks (CNNs or ConvNet) are deep machine learning algorithms that use multiple weighted hidden layers of convolutional filters to make decisions about a given input. Recent developments in general-purpose computing on graphics processing units (GPGPU) have made the use of ConvNets practical, and as a result ConvNets have since been shown to be particularly well suited for the task of classifying and analysing images \citep{ciresan2011, ciresan2012}. Neural networks have the advantage of dynamically ``learning'' and optimising features, as opposed to relying on human-selected features. This manner of feature selection allows for the potential discovery of new underlying concepts that fundamentally define a class from others, and also prevents assumptions and misconceptions from biasing features and introducing inaccuracies.\par

%NOTE: Might be better suited for Discussion section
The automated nature of feature selection in neural networks can also pose challenges if the dataset is of low-quality or very small; as differences between classes that are present in the dataset, but do not reflect differences between the classes as a concept, can severely bias a model. This is epitomised by ``the tank problem'', whereby a neural network trained by the US government to identify concealed tanks were actually identifying dark skies as the images of tanks in the dataset were taken on a cloudy day \citep{dreyfus1992}. Errors such as these are mitigated by assuring datasets are large and representative of the variation of the population.\par

ConvNets have been successfully used to create very accurate models for the classification of breast cancer lesions. Binary models for classifying benign and malignant lesions, as well as multi-class models for distinguishing between multiple subtypes of breast lesions from H\&E stained biopsy slides have established with very high ($>90\%$) accuracy \citep{wei2017, han2017}.  
